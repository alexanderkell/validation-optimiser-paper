%% 
%% Copyright 2007-2019 Elsevier Ltd
%% 
%% This file is part of the 'Elsarticle Bundle'.
%% ---------------------------------------------
%% 
%% It may be distributed under the conditions of the LaTeX Project Public
%% License, either version 1.2 of this license or (at your option) any
%% later version.  The latest version of this license is in
%%    http://www.latex-project.org/lppl.txt
%% and version 1.2 or later is part of all distributions of LaTeX
%% version 1999/12/01 or later.
%% 
%% The list of all files belonging to the 'Elsarticle Bundle' is
%% given in the file `manifest.txt'.
%% 
%% Template article for Elsevier's document class `elsarticle'
%% with harvard style bibliographic references

%\documentclass[preprint,12pt,authoryear]{elsarticle}

%% Use the option review to obtain double line spacing
%% \documentclass[authoryear,preprint,review,12pt]{elsarticle}

%% Use the options 1p,twocolumn; 3p; 3p,twocolumn; 5p; or 5p,twocolumn
%% for a journal layout:
%% \documentclass[final,1p,times,authoryear]{elsarticle}
%% \documentclass[final,1p,times,twocolumn,authoryear]{elsarticle}
%% \documentclass[final,3p,times,authoryear]{elsarticle}
\documentclass[final,3p,times,twocolumn,numbers]{elsarticle}
%% \documentclass[final,5p,times,authoryear]{elsarticle}
%% \documentclass[final,5p,times,twocolumn,authoryear]{elsarticle}

%% For including figures, graphicx.sty has been loaded in
%% elsarticle.cls. If you prefer to use the old commands
%% please give \usepackage{epsfig}

%% The amssymb package provides various useful mathematical symbols
\usepackage{amssymb}
%% The amsthm package provides extended theorem environments
\usepackage{amsthm}

\usepackage{mhchem}
\usepackage{xcolor}
\usepackage{csvsimple,booktabs}

%% The lineno packages adds line numbers. Start line numbering with
%% \begin{linenumbers}, end it with \end{linenumbers}. Or switch it on
%% for the whole article with \linenumbers.
%% \usepackage{lineno}

\journal{Applied Energy}

\begin{document}

\begin{frontmatter}

%% Title, authors and addresses

%% use the tnoteref command within \title for footnotes;
%% use the tnotetext command for theassociated footnote;
%% use the fnref command within \author or \address for footnotes;
%% use the fntext command for theassociated footnote;
%% use the corref command within \author for corresponding author footnotes;
%% use the cortext command for theassociated footnote;
%% use the ead command for the email address,
%% and the form \ead[url] for the home page:
 \title{Validating the long-term electricity market model ElecSim using genetic algorithms}
% \tnotetext[label1]{}
 \author{Alexander J. M. Kell}
 \ead{a.kell2@newcastle.ac.uk}
% \ead[url]{home page}
% \fntext[label2]{}
% \cortext[cor1]{}
% \address{Address\fnref{label3}}
% \fntext[label3]{}

%\title{Validating a long-term electricity market model}

%% use optional labels to link authors explicitly to addresses:
%% \author[label1,label2]{}
%% \address[label1]{}
%% \address[label2]{}

\author{A. Stephen McGough, Matthew Forshaw}

\address{School of Computing, Newcastle University, Newcastle-upon-Tyne, United Kingdom}

\begin{abstract}
%% Text of abstract

\end{abstract}
%
%%%Graphical abstract
%\begin{graphicalabstract}
%\includegraphics{grabs}
%Hello test
%\end{graphicalabstract}
%
%%%Research highlights
%\begin{highlights}
%\item Validating a model
%\item Optimisation
%\item Scenario modelling
%\end{highlights}

\begin{keyword}
%% keywords here, in the form: keyword \sep keyword
Long-term energy modelling \sep model validation \sep Machine learning \sep Optimization
%% PACS codes here, in the form: \PACS code \sep code

%% MSC codes here, in the form: \MSC code \sep code
%% or \MSC[2008] code \sep code (2000 is the default)

\end{keyword}

\end{frontmatter}

%% \linenumbers

%% main text
\section{Introduction}
\label{sec:intro}


To limit the effects of climate change, a transition from a fossil-fuel based energy system to one based on low-carbon, renewable energy is required. The report by the Intergovernmental Panel on Climate Change detailed that reaching and sustaining zero global anthropogenic \ce{CO2} would halt anthropogenic global warming on multi-decadal time scales \cite{Masson-Delmotte2018}. 

The Paris Agreement was a declaration signed in 2015 by 195 state parties to plan and regularly report on the contribution made to mitigate global warming \cite{May2002}. Based on this commitment, policy makers require quantitive advice on interventions to aid in the mitigation of climate change and limit global average temperatures to well below 2$^{\circ}$C. 

The decarbonization of electricity generation is of strategic importance for this goal due to the fact that low-carbon electricity can enable reductions in \ce{CO2} emissions in industry, transport and building sectors~\cite{Salas2017}. 

However, there remain a number of uncertainties in the technological transition to a low-carbon energy supply. Examples of these uncertainties are investor behaviour, future prices of electricity generation and storage, domestic and international policy, energy efficiency and electricity demand. To successfully create effective policies an increase in understanding of these uncertainties and how they interact is required.

Energy modelling is a method that allows policy makers to increase their understanding of policy decision outcomes under a wide range of scenarios. Agent-based modelling (ABM) is a simulation technique that allows for heterogeneous agents to interact and can lead to effects on the aggregated level of the total system, a phenomenon called ``emergence'' \cite{EpsteinJoshuaM.author.GSSS}. Traditional models for analysing electricity systems, such as centralised optimisation models do not account for the heterogeneous nature of electricity investors and are, to some extent, based on obsolete assumptions~\cite{Ringler2016}.

In this paper we motivate that agent-based models are a valid way of complimenting existing models to provide advice to decision makers. We show that the model ElecSim \cite{Kell} can be validated over a 5 year period, starting from the year 2013 and ending in the year 2018, with a root mean squared error of {\color{red} $\sim0.045$} and a standard deviation of {\color{red}$\sim0.16$}. Similarly to Nahmmacher \textit{et al.} we demonstrate how clustering of multiple relevant time series such as electricity demand, solar irradiance and wind speed can reduce computational time by selecting representative days~\cite{Nahmmacher2016}. However, distinctly to Nahmacher \textit{et al.} we use a K-means clustering approach \cite{forgy65} as opposed to a hierarchical clustering algorithm described by Ward \cite{doi:10.1080/01621459.1963.10500845}.

We use a genetic algorithm approach to find an optimal set of price curves predicted by generation companies (GenCos) that  adequately model observed investment behaviour in the real-life electricity market in the United Kingdom. However, similar techniques can be employed for other countries of various sizes \cite{Kell}. We are able to model the transitional dynamics of the electricity mix in the United Kingdom as shown in Figure \ref{uk_historical_mix}, where there was an $\sim88\%$ drop in coal use, $\sim44\%$ increase in Combined Cycle Gas Turbines (CCGT), $\sim111\% $ increase in wind energy and increase in solar from near zero to $\sim 1250$MW.


\begin{figure}
\centering
\includegraphics[width=0.46\textwidth]{figures/introduction/uk_historical_mix.pdf}
\label{uk_historical_mix}
\caption{Electricity generation transition from 2013 to 2018 in the United Kingdom.}
\end{figure}

There is a desire to validate the ability of energy-models to make long-term predictions. Validation increases confidence in the outputs of a model and leads to an increase in trust amongst the public and policy makers. However, energy models are frequently criticised for being insufficiently validated, with the performance of models rarely checked against historical outcomes \cite{Beckman2011}.

The model OSeMOSYS \cite{Howells2011}, however, is validated against the similar model MARKAL\slash TIMES.  Whereas the model PowerACE shows that realistic prices are achieved through modelling, however do not indicate success in modelling investor behaviour \cite{Ringler2012}.

However, under the definition by Hodges \textit{et al.} \cite{Hodges} long-range energy forecasts are not validatable \cite{Craig2002}. Under this definition, validatable models must be observable, exhibit constancy of structure in time, exhibit constancy across variations in conditions not specified in the model and it must be possible to collect ample data \cite{Hodges}.

Whilst it is possible to collect data for energy models, the data covering important characteristics of energy markets are not always measured. Furthermore, the behaviour of the human population and innovation are neither constant or entirely predictable. This leads to the fact that static models cannot keep pace with global long-term evolution. Assumptions made by the modeller may be challenged in the form of unpredictable events, such as the oil shock of 1973.

This, however, does not mean that energy-modelling is not useful for providing advice in the present. A model may fail at predicting the long-term future because it has forecast an undesirable event, which lead to a change in human behaviour. Thus avoiding the original scenario that was predicted.

A retrospective study published in 2002 by Craig \textit{et al.} focused on the ability for forecasters to accurately predict electricity demand from the 1970s \cite{Craig2002}. They found that actual energy usage in 2000 was at the very lowest end of the forecasts, with only one exception. They found that these forecasts underestimated unmodelled shocks such as the oil crises which lead to increased energy efficiency.

Hoffman \textit{et al.} also developed a retrospective validation of a predecessor of the current MARKAL\slash TIMES model named Reference Energy System \cite{Hoffman_1973}, and the Brookhaven Energy System Optimization Model \cite{ERDA_48}. These were studies applied in the 70s and 80s to develop projections to the year 2000 . They found that the models were able to be descriptive, but were not entirely accurate in terms of predictive ability. They found that emergent behaviours in response to policy had a strong impact on forecasting accuracy. They concluded that forecasts must be expressed in highly conditioned terms \cite{Hoffman2011}. 

Schurr \textit{et al.} argued against predicting too far ahead in energy modelling due to the uncertainties involved \cite{Schurr_1961}. However, they specify that long-term energy forecasting is useful to provide basic information on energy consumption and availability which is helpful in public debate and in guiding policy makers.


Ascher concurs with this view, and states that the most significant factor in model accuracy is the time horizon of the forecast, with the more distant the forecast target, the less accurate, due to unforeseen changes in society as a whole ~\cite{gillespie_1979}.

Work by Koomey \textit{et al.} expresses the importance of conducting retrospective studies to help improve models \cite{Koomey2003}. For example, a model can be rerun using historical data in order to determine how much of the error in the original forecast resulted from structural problems in the model itself and how much from incorrect specification of the fundamental drivers of the forecast \cite{Koomey2003}.


It is for the reasons previously described that this paper focuses on a shorter-term (5 year) horizon window when validating the model. This enabled us to increase confidence that the dynamics of the model worked without external shocks and could provide descriptive advice to stakeholders.


%\begin{itemize}
%	\item Energy systems modelling to help transition to low-carbon energy systems (Paris Agreement)
%	\item Application of quantitive analysis to policy
%	\item Use of agent-based models to model heterogeneous actors
%	\item Optimum policy interventions for a smooth transition
%	\item Requirement to validate model using historical data
%	\item Prediction of electricity prices to understand optimal decisions
%	\item Confidence in model under certain scenarios
%\end{itemize}



\section{Material and methods}
\label{sec:methods}

The model, ElecSim, is made up of five distinct sections: power plant data; scenario data; the time-steps of the algorithm; the power exchange and the investment algorithm. ElecSim has been previously published \cite{Kell}, however, amendments have since been made to the model in the form of efficiency improvements as well as increasing the granularity of time-steps from yearly to representative days. In this paper we have used 8 representative days for electricity demand, solar irradiance and offshore and onshore wind speed.

In this section we summarize previously published results and detail the modifications made for this paper. In this paper we initialised the model to a scenario of the United Kingdom, however, the model is generalisable to any country and is dependent on input data.

\subsection{Plant Data}

The simulation was initialised with every power plant and generation company (GenCo) in the United Kingdom using the Department of Business, Energy and Industrial strategy of the British government for the year 2013 \cite{dukes_511}. The individual costs of these power plants were also initialised with data from the Department of Business, Energy and Industrial strategy of the British government \cite{Department2016}. For power plants that were out of the scope of this dataset (pre-2018), historical levelized cost of electricity (LCOE) values were used to infer the granular costs by a scaling factor from the International Energy Agency (IEA)~\cite{IEA2015}. Historical efficiency was also taken from the Energy Information Administration (EIA) \cite{eia_efficiency}. Each of the initialised power plants were initialized with a scaling factor which modified the operation and maintenance costs stated in the dataset provided \cite{Department2016}. This was done to take into account differences in labor, land and breakages between projects. This was sampled from a uniform distribution between 0.3 and 2.0. As well as varying operation and maintenance costs, each of the GenCos purchased fuel at varying prices. This was done to model the element of chance and differing hedging strategies of each of the GenCos. The distribution that this was sampled from was taken from fitting an ARIMA model and sampling from the standard deviation of the residuals \cite{ARIMA}.

Financing of the project was provided by stock shares and debt, with nuclear power plants given an average weighted average cost of capital (WACC) of $0.1$ and non-nuclear given a WACC of $0.059$ \cite{KPMG2017, Paper2012}. Where WACC is the rate that a company is expected to pay on average for its stock and debt.

\subsection{Power Exchange}

ElecSim is modelled on a uniform pricing market. This means that bids are sorted with respect to price, and accepted in merit-order. Merit-order in this case indicates the cheapest bids are accepted first. Uniform pricing is a market mechanism in which the highest accepted bid price is paid to all generators irrespective of price bid. This mechanism encourages GenCos to bid their short run marginal cost (SRMC) to ensure that their power plant is dispatched whilst not losing money whilst dispatching.

In the case of ElecSim SRMC is defined as follows:
\begin{equation}
\label{eq:srmc}
	SRMC = O\&M_{var}+CO_{2price}+\left(fuel_{price}\times mod_{fuel}\right)
\end{equation}

Where $O\&M_{var}$ is the variable operating and maintenance costs, $CO_{2price}$ is the carbon tax, $fuel_{price}$ is the cost of the respective fuel and $mod_{fuel}$ is the fuel price modifier. These are all in the units of  $\textsterling\slash MWh$. 

Each of the GenCos submit a bid based on the SRMC of each of their plants at the start of every representative day, to model the day-ahead market. 
\begin{equation}
	Bid = SRMC\times Cap_{fac} \times Avail_{fac}
\end{equation}

Where $Cap_{fac}$ and $Avail_{fac}$ is the capacity and availability factors of the plants respectively. The capacity factor is defined as the actual electrical energy produced over a given time period divided by the maximum possible electrical energy is could have produced. This can be impacted by regulatory constraints, market forces and resource availability. For example, higher capacity factors are common for photovoltaics in the summer, and lower in winter \cite{Kell}.

\begin{equation}
	Cap_{fac}=\frac{energy_{produced}}{energy_{max}}
\end{equation}

 Availability is the percentage of time that a power plant is able to produce electricity. This is typically reduced by outages and breakdowns. We integrate historical data to model improvements in reliability over time.
 
 This uniform pricing market is stepped for every representative day, which, in the case of this model was 8 days. 
 
 \subsection{Investment Algorithm}
 
Investments in power plants occur at the beginning of each year. Each GenCo sequentially assesses the viability of different power plants. The order of GenCo is randomized every year to prevent certain GenCos having an advantage over others.

Investment in power plants is based upon a net present value (NPV) calculation. NPV is a summation of the present value of a series of present and future cash flow. This metric provides a method for evaluating and comparing investments with cash flows that are spread over many years. 

Equation \ref{eq:npv_eq} is the calculation of NPV, where $t$ is the year of the cash flow, $i$ is the discount rate, $N$ is total number of periods, or lifetime of power plant, and $R_t$ is the net cash flow at time $t$.
\begin{equation} \label{eq:npv_eq}
NPV(i, N) = \sum_{t=0}^{N}\frac{R_t}{(1+t)^t}
\end{equation}

The discount rate set by the GenCo is based upon the WACC \cite{KincheloeStephenC1990TWAC}. We sample from a Gaussian distribution to adjust to adjust for varying risk profiles, opportunity costs and rates of return. Giving us sufficient variance whilst deviating from the expected price.

Future cash flow is based upon predicted earnings, which is based upon a predicted, exogenous price duration curve (PDC). A PDC is the cost of electricity with respect to hours of the year. A central part of this paper is on estimating a suitable predicted PDC that enables us to validate the model.

The SRMC cost of the power plant is calculated by fitting a linear regression to historical \ce{CO2} and fuel price, referred to in Equation \ref{eq:srmc}. 

The plant with the highest NPV is chosen by each of the GenCos.
 
 
\subsection{Representative days}

Due to computational restrictions, energy-system models often represent variations in demand, supply, solar irradiance and wind speed by using the data of a limited number of representative historical days \cite{Poncelet2017}.

A number of authors have shown that models with an insufficient time-step granularity leads to an underestimation of the variability of intermittent renewable energy resources (IRES). This leads to an overestimation of the uptake of IRES and an underestimation of flexible technologies~\cite{Ludig2011,Haydt2011}. We exhibit the same problems in our paper \cite{Kell} and in Section \ref{sec:results}.


\begin{figure*}
\centering
\includegraphics[width=\textwidth]{figures/methods_and_materials/clusters_compared.pdf}
\label{cluseters_compared}
\caption{Comparison of number of clusters for accuracy.}
\end{figure*}
 
\subsection{Problem Formulation}




\begin{itemize}
	\item Reproducible data
	\item Summarize previously published results
	\item Modifications of previous results for this paper
\end{itemize}

\section{Calculations}
\label{sec:calculations}

\begin{itemize}
	\item Practical development
\end{itemize}

\section{Results}
\label{sec:results}

\begin{table}[htb]
    \centering
\csvautobooktabular{table_data/results/error_metrics.csv}
    \caption{Error metrics for time series forecast from 2013 to 2018}\label{table:metrics}
\end{table}


\begin{figure}
\centering
\includegraphics[width=0.49\textwidth]{figures/results/throughout_years.pdf}
\label{uk_historical_mix}
\caption{Electricity generation transition from 2013 to 2018 in the United Kingdom.}
\end{figure}


\begin{figure}
\centering
\includegraphics[width=0.46\textwidth]{figures/results/best_run.pdf}
\label{uk_historical_mix}
\caption{Electricity generation transition from 2013 to 2018 in the United Kingdom.}
\end{figure}

\begin{itemize}
	\item Clear and concise results
\end{itemize}

\section{Discussion}
\label{sec:discussion}

\begin{itemize}
	\item Significance of work
	\item Avoid discussion of public work
\end{itemize}

\section{Conclusion}
\label{sec:conclusion}

\begin{itemize}
	\item Main conclusions
\end{itemize}

\section{Funding Sources}

This work was supported by the Engineering and Physical Sciences Research Council, Centre for Doctoral Training in Cloud Computing for Big Data [grant number EP/L015358/1].




%% The Appendices part is started with the command \appendix;
%% appendix sections are then done as normal sections
%% \appendix

%% \section{}
%% \label{}

%% If you have bibdatabase file and want bibtex to generate the
%% bibitems, please use
%%
  \bibliographystyle{elsarticle-num} 
  \section*{References}
  \bibliography{library,bib_custom}

%% else use the following coding to input the bibitems directly in the
%% TeX file.

%\begin{thebibliography}{00}
%
%%% \bibitem[Author(year)]{label}
%%% Text of bibliographic item
%
%\bibitem[ ()]{}
%
%\end{thebibliography}
\end{document}

\endinput
%%
%% End of file `elsarticle-template-harv.tex'.
	